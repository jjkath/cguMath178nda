% !TEX spellcheck = en-US
\chapter{Concluding Remarks}
\label{cha:conclusions} 
The goal of this tutorial was not to give a complete overview of all algorithms that can be used for position and orientation estimation. Instead, our aim was to give a pedagogical introduction to the topic of position and orientation estimation using inertial sensors, allowing newcomers to this problem to get up to speed as fast as possible by reading one paper. By integrating the inertial sensor measurements (so-called dead-reckoning), it is possible to obtain information about the position and orientation of the sensor. However, errors in the measurements will accumulate and the estimates will drift. Because of this, to obtain accurate position and orientation estimates using inertial measurements, it is necessary to use additional sensors and additional models. In this tutorial, we have considered two separate estimation problems. The first is orientation estimation using inertial and magnetometer measurements, assuming that the acceleration of the sensor is approximately zero. Including magnetometer measurements removes the drift in the heading direction (as illustrated in \Exampleref{ex:oriEst-noMagData}), while assuming that the acceleration is approximately zero removes the drift in the inclination. The second estimation problem that we have considered is pose estimation using inertial and position measurements. Using inertial measurements, the position and orientation estimates are coupled and the position measurements therefore provide information also about the orientation. 

A number of algorithms for position and orientation estimation have been introduced in \Chapterref{cha:orientationEstimation}. These include smoothing and filtering solved as an optimization problem, two different extended Kalman filter implementations and a version of the complementary filter. The filtering approaches use the data up to a certain time $t$ to estimate the position and orientation at this time $t$. Smoothing instead makes use of all data from time $t = 1, \hdots , N$. In general, using all data to obtain the estimates will naturally lead to better estimates. The filtering approaches can be seen to be quite uncertain about their estimates for the first samples and require some time to ``converge'' to accurate estimates. This is even more pronounced in the presence of calibration parameters as discussed in \Chapterref{cha:calibration}. Although smoothing algorithms give better estimates, practical applications might not allow for computing smoothing estimates because of computational limitations or real-time requirements. For the examples discussed in this paper, the optimization-based filtering algorithm and the \gls{ekf} with orientation deviation states perform very similarly. The \gls{ekf} with quaternion states, however, was able to handle wrong initial orientations less well as shown in \Tableref{tab:oriEst-rmsSimWrongInit}. Furthermore, it underestimated the uncertainty in the heading direction in the absence of magnetometer measurements, see also \Exampleref{ex:oriEst-noMagDataCovComp}. The complementary filter is a good alternative to these methods, but its estimation accuracy decreases when orientation estimates from the accelerometer and magnetometer measurements are of significantly better quality in the inclination than in the heading direction. In \Chapterref{cha:applications} we have discussed possible extensions to the simple models considered earlier in this tutorial. One of the benefits of the optimization-based approaches is that these extensions can straightforwardly be incorporated into the framework. 

Apart from the differences between the estimation algorithms discussed in this tutorial, it can also be concluded that the position and orientation estimation problems using inertial sensors are actually quite forgiving. Any of the algorithms introduced in this tutorial can give reasonably good estimates with fairly little effort. However, careful modeling is important since the quality of the estimates of the algorithms highly depends on the validity of the models. This was illustrated in \Exampleref{ex:oriEst-magDist} for the influence of magnetic material in the vicinity of the sensor on the quality of the orientation estimates. In recent years, inertial sensors have undergone major developments. The quality of their measurements has improved while their cost has decreased, leading to an increase in availability. Furthermore, available computational resources are steadily increasing. Because of these reasons, we believe that inertial sensors can be used for even more diverse applications in the future. 