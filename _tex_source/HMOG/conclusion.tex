%
\section{Conclusion and Future Work}
\label{sec:conclusion}

In this paper, we introduced HMOG, a set of behavioral biometric features for continuous 
authentication of smartphone users.  We evaluated HMOG from three perspectives---continuous authentication, BKG, and energy consumption. Our evaluation was performed on multi-session data collected from 100 subjects under two motion conditions (i.e., sitting and walking). Results of our evaluation can be summarized as follows. By combining HMOG with tap features, we achieved 8.53\% authentication EER during walking and 11.41\% during sitting, which is lower than the EERs achieved individually with tap or HMOG features. Further, by fusing HMOG, tap and keystroke dynamic features, we achieved the lowest EERs (7.16\% in walking and 10.05\% in sitting). Our results demonstrate that HMOG is well suited for continuous authentication of smartphone users. In fact, HMOG improves the performance of taps and keystroke dynamic features, especially during walking---a common smartphone usage scenario.  For BKG, HMOG features provide lower EER (17.4\%) compared to tap (25.7\%) and swipe features (34.2\%). Moreover, fusion of HMOG with tap features provide the best performance, with 15.1\% EER. Additionally, the energy overhead of sample collection and feature extraction is small (less than 8\% energy overhead when sensors were sampled at 16Hz). This makes HMOG well suited for energy-constrained devices such as smartphones. 
%



%





%
%
%
%
%
%
As future work, we plan to investigate how HMOG features perform under stringent constraints such as: (a) walking at higher speeds; (b) using the smartphone in different weather conditions; and (c) using applications that do not involve typing (e.g., browsing a map). Another research question of interest is \textit{cross-device interoperability}, i.e., how and to what extent can a user's behavioral biometric collected on a desktop (e.g., keystroke dynamics) be leveraged with HMOG features to authenticate the user on a smartphone (and vice versa). 
%