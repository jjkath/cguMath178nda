%
\newpage
\appendices


\section{Security analysis of our BKG Scheme}
\label{sec:security}
%

%

We prove that our BKG technique meets the requirements from \cite{bal08}---namely, that 
cryptographic keys are indistinguishable from random 
given the commitment ({\em key randomness}), and that given a cryptographic key and a commitment, no 
useful information about the biometric can be reconstructed ({\em biometric privacy}).
We assume that the biometric is modeled by an unpredictable function. This captures the idea 
that a user's biometric is {\em difficult to guess}. Informally, an 
unpredictable function $f(\cdot)$ is a function for which no efficient adversary 
can predict $f(x^*)$ given $f(x_i)$ for various $x_i \neq x^*$. Formally:

\begin{definition}
A function family $(\C, D, R, F)$ for $\{f_c(\cdot): D \rightarrow~R\}_{c \leftarrow \C}$ is unpredictable if for any efficient algorithm $\A$ and auxiliary information $z$ we have:
\[
	Pr[(x^*, f_c(x^*) \leftarrow \A^{f_c(\cdot)}(z) \text{\; and\; } x^* \not\in Q] \leq \operatorname{negl}(\kappa)
\]
where $Q$ is the set of queries from $\A$, $\kappa$ is the security parameter and $\operatorname{negl}(\cdot)$ is a negligible function.
\label{def:unpredictable}
\end{definition}

In order to define security of biometric key generation systems, Ballard et al.~\cite{bal08} introduced the notions of \emph{Key Randomness} (REQ-KR), \emph{Weak Biometric Privacy} (REQ-WBP) and \emph{Strong Biometric Privacy} (REQ-SBP). 
%
%
We formalize the notion of key randomness by defining Experiment $\operatorname{IND-KR}_\A(\kappa)$:

\medskip
\noindent 
{\bf Experiment $\operatorname{IND-KR}_\A(\kappa)$} 

\smallskip
\begin{enumerate}
	\item $\A$ is provided with a challenge $(\PRF_{c_i}(z|1), \delta)$, $k_b$ and $z$, where $k_0 = \PRF_{c_i}(z|0)$ and $k_1 \leftarrow_R \{0,1\}^\kappa$ for a bit $b \leftarrow_R \{0,1\}$, corresponding to user $i$.
	\item $\A$ is allowed to obtain biometric information $x_j$ for arbitrary users $j$ such that $j\neq i$.
	\item $\A$ outputs a bit $b'$ as its guess for $b$. The experiment outputs $1$ if $b = b'$, and $0$ otherwise. 
\end{enumerate}

\begin{definition}
We say that a biometric key generation scheme has the Key Randomness property if there exist a negligible function $\operatorname{negl}(\cdot)$ such that for any PPT $\A$, $\operatorname{Pr}[\operatorname{IND-KR}_\A(\kappa) = 1] \leq 1/2+\operatorname{negl}(\kappa)$.
\end{definition}

\begin{theorem} \label{thm:key_randomness}
Assuming that the $\PRF$ is a pseudo-random function family and that biometric $x = (x_1, ..., x_n)$ is unpredictable, our Fuzzy Commitment scheme has the Key Randomness property.
\label{thm:REQ-KR}
\end{theorem}

\begin{proof}[Proof of Theorem \ref{thm:key_randomness} (Sketch)]
%
Because $c = x - \delta$, and $x$ is assumed to be unpredictable, $c$ is unpredictable given $\delta$. We now show that any PPT adversary $\A$ that has advantage $1/2+\Delta(\kappa)$ to win the $\operatorname{IND-KR_\A(\kappa)}$ experiment can be used to construct a distinguisher $\D$ that has similar advantage in distinguishing $\PRF$ from a family of uniformly distributed random functions.

$\D$ is given access to oracle $O(\cdot)$ that selects a random codeword $c$ and a random bit $b$, and responds to a query $q$ with random (consistent) values if $b=1$, and with $\PRF_c(q)$ if $b=0$. 
$\D$ selects a random $z$, a codeword $c'$ and a feature vector $x'$, and sets $\delta' = x' - c'$. Then $\D$ sends $\gamma' = (O(z|1), \delta')$ and $O(z|0)$ to $\A$. %

%
If $b=0$, then pair $(\gamma', \PRF_{c'}(z|0))$ is indistinguishable from $((\PRF_{c}(z|1), \delta), \PRF_{c}(z|0))$, because $\delta$ and $\delta'$ follow the same distribution, $c$ and $c'$ are unpredictable given $\delta$ and thus both $\PRF_{c}(\cdot)$ and $\PRF_{c'}(\cdot)$ are indistinguishable from random. 
If $b=1$, then $O(\cdot)$ is a random oracle, so $(\gamma', O(z))$ is 
indistinguishable from pair $((\PRF_c(z|1), \delta), \PRF_c(z|0))$: $c$ is 
unpredictable given $\delta$,  therefore $\PRF_c(\cdot)$ is indistinguishable from 
random. 
 
Eventually, $\A$ outputs $b'$, and $\D$ outputs  $b'$ as its guess. It is easy to see that $\D$ wins iff $\A$ wins, so $\D$ is correct with probability $1/2+\Delta(\kappa)$. Therefore,  $\Delta(\cdot)$ must be a negligible function.\end{proof}

REQ-WBP states that the adversary learns no useful information about a biometric 
signal from the commitment and the auxiliary information, while REQ-SBP states 
that the adversary learns no useful information about the biometric given 
auxiliary information, the commitment and the key.
For our BKG algorithms, REQ-SBP implies REQ-WBP. In fact, $\PRF_c(z|1)$ (which is part of the commitment) is known to the adversary, and therefore $k=\PRF_c(z|0)$ does not reveal any additional information. %
%
%
From the unpredictability of $x$, it follows that the output of $\PRF_c$ does not reveal $c$, so $\PRF_c(z|1)$ and $k$ do not disclose information about $x$. 

