\documentclass{article}

% if you need to pass options to natbib, use, e.g.:
% \PassOptionsToPackage{numbers, compress}{natbib}
% before loading nips_2016
%
% to avoid loading the natbib package, add option nonatbib:
% \usepackage[nonatbib]{nips_2016}

% \usepackage{nips_2016}

% to compile a camera-ready version, add the [final] option, e.g.:
\usepackage[final]{nips_2016}

\usepackage[utf8]{inputenc} % allow utf-8 input
\usepackage[T1]{fontenc}    % use 8-bit T1 fonts
\usepackage{hyperref}       % hyperlinks
\usepackage{url}            % simple URL typesetting
\usepackage{booktabs}       % professional-quality tables
\usepackage{amsfonts}       % blackboard math symbols
\usepackage{nicefrac}       % compact symbols for 1/2, etc.
\usepackage{microtype}      % microtypography

\title{Reading Summary 1 - HMOG: New Behavioral Biometric Features for Continuous Authentication of Smartphone Users}

\author{
  John Kath \\
  \texttt{john.kath@cgu.edu} \\
}

\begin{document}

\maketitle

% \begin{abstract}
% \end{abstract}

\section{Project Overview}

\subsection{Description}

\textbf{Project Type (1)}

\begin{enumerate}
  \item Application of existing algorithm to a new problem and
    potentially new data.
\end{enumerate}

\subsection{Requirements}

\begin{itemize}
  \item \textbf{Partner:} Work independently.
  \item \textbf{Dataset:} HMOG cell phone data from at least five
    users/activities. Will apply data analytics and ML methods using
    accelerometer, gyroscope and magnetometer data (3-axis each).
  \item \textbf{Format:} \LaTeX{} NIPS
  \item \textbf{Code Style:} Will use suggested code style guidelines
    (cookiecutter data science) with MIT open-source license.
  \item \textbf{Programming Tools \& Hardware:} Python/Jupyter
    Notebook, C++, NVIDIA Jetson Nano with Jetpack API.
\end{itemize}

\subsection{Proposed Project}

I would like to use the probabilistic models and data analytics methods
introduced in the paper [1] \textit{Using Inertial Sensors for Position
and Orientation Estimation} to estimate the position and orientation
(pose) of a users cell phone during an activity. Cell phone 3D
accelerometer, gyroscope and magnetometer data will be obtained for
analysis from the HMOG dataset associated with the article [2]
\textit{HMOG: New Behavioral Biometric Features for Continuous
Authentication of Smartphone Users}.

\textbf{Here are the proposed techniques for data analytics discussed
in [1] \textit{Using Inertial Sensors for Pose Estimation}:}

\begin{itemize}
  \item \textbf{(Ch2) Inertial Sensors:} Coordinate frames, angular
    velocity, specific force, sensor error.
  \item \textbf{(Ch3) Probabilistic Models:} Parameterizing/probabilistic
    orientation modeling (Euler angles, Unit quaternions),
    measurement/probabilistic models for pose estimation.
  \item \textbf{(Ch4) Estimating Position and Orientation:} Smoothing
    in optimized frame (Gauss-Newton estimation, uncertainty),
    smoothing estimation of orientation using optimization, filtering
    estimate of orientation using optimization, filtering estimate in
    optimization framework, extended Kalman filtering / complementary
    filtering.
\end{itemize}

\section*{References}

\small

[1] SITOVÁ, Zdeňka, Jaroslav ŠEDĚNKA, Qing YANG, Ge PENG, Gang ZHOU,
Paolo GASTI and Kiran BALAGANI. HMOG: New Behavioral Biometric Features
for Continuous Authentication of Smartphone Users. {\it IEEE Transactions on
Information Forensics and Security, 2016, Vol. 11, No. 5}, p. 877 - 892.
ISSN 1556-6013. \\ \url{http://dx.doi.org/10.1109/TIFS.2015.2506542}

\end{document}
